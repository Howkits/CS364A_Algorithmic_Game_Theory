\documentclass{article}
\usepackage{graphicx,fancyhdr,amsmath,amssymb,amsthm,subfig,url,hyperref}
\usepackage[margin=1in]{geometry}
\usepackage[UTF8]{ctex}

%----------------------- Macros and Definitions --------------------------

%%% FILL THIS OUT
\newcommand{\studentname}{NT}
\newcommand{\suid}{123}
\newcommand{\exerciseset}{Exercise Set 1}
%%% END



\renewcommand{\theenumi}{\bf \Alph{enumi}}

%\theoremstyle{plain}
%\newtheorem{theorem}{Theorem}
%\newtheorem{lemma}[theorem]{Lemma}

\fancypagestyle{plain}{}
\pagestyle{fancy}
\fancyhf{}
\fancyhead[RO,LE]{\sffamily\bfseries\large Stanford University}
\fancyhead[LO,RE]{\sffamily\bfseries\large CS 364A Algorithmic Game Theory}
\fancyfoot[LO,RE]{\sffamily\bfseries\large \studentname: \suid @stanford.edu}
\fancyfoot[RO,LE]{\sffamily\bfseries\thepage}
\renewcommand{\headrulewidth}{1pt}
\renewcommand{\footrulewidth}{1pt}

\graphicspath{{figures/}}

%-------------------------------- Title ----------------------------------

\title{CS364A \exerciseset}
\author{\studentname \qquad SUNet ID: \suid}

%--------------------------------- Text ----------------------------------

\begin{document}
\maketitle

\section*{Lecture 1}
\section*{Problem 1}
Give at least two suggestions for how to modify the Olympic badminton tournament format to reduce or
eliminate the incentive for teams to intentionally lose a match.

问题的根源在于因为较少数目的比赛导致了爆冷场次导致的的比赛者排名与实力的不匹配,原本有利于水平较高的选手的比赛
机制适得其反,因此将优化目标放在减少这种小概率事件导致的后果影响上。
\begin{enumerate}
\item %A
增加组内人员数目,水平更高的选手排名将更靠前,同时增加每个组的晋级人数

\item %B
增加组内循环比赛的轮数,选手之间比赛次数更多,胜负关系不由某1场的结果决定

\item %C
修改比赛规则,小组赛后增加胜者组和败者组,根据比赛排名决定组别,使用双败淘汰赛制,胜者组队伍输掉比赛后仍然可以在败者组
进行比赛,败者组队伍输掉比赛直接淘汰

\end{enumerate}

\section*{Problem 2}
For this exercise and the next, see Section 1.3 of the AGT book (available for free from the course Web site)
for a formal definition of a Nash equilibrium.
\begin{enumerate}
\item %A
Watch the scene from A Beautiful Mind that purports to explain what a Nash equilibrium is. (It’s
easy to find on YouTube.) The scenario described is most easily modeled as a game with four players
(the men), each with the same five actions (the women). Explain why the solution proposed by the
John Nash character (i.e., Russell Crowe) is not a Nash equilibrium.

解答:首先看纳什均衡的定义,在两个以上参与者在知道其他人的均衡策略后没有人可以在其他人获益情况不变的情况
下增加自己的收益,同时保证收益最大化,强调的是解的稳定性。电影中主角的解显然是不稳定的,四个人中的任何
一个人都可以改变自己的策略选择约会金发女,此时在其他人选择不变的情况下可以将自己的收益最大化。此时这个
解是纳什均衡解。
\item %B
\textbf{(Optional extra credit)} Propose a plausible game-theoretic model (probably a sequential game,
rather than a single-shot game) in which the solution proposed by Nash/Crowe is a Nash equilibrium.

解答:在贯序博弈模型中,每个人的策略将基于前一个人制定,不妨设置这样一种机制,除第一个人外,每个人做出选择
后必须和前一个人的选择进行交换,在这种情况下,Nash/Crowe的方案就是一个纳什均衡解,因为每个人都不能做到在其
他人收益不变的情况下通过改变自己的策略增加收益,举个例子,在这种情况下每个人收益设为1,选择金发女收益为2,当
其中一个人改变策略选择金发女的时候,无论如何都不能将自己的收益增加,因此是一个纳什均衡解,另外当有任何一个
人的策略是选择金发女时,同样是纳什均衡解
\end{enumerate}

\section*{Problem 3}
Prove that there is a unique (mixed-strategy) Nash equilibrium in the Rock-Paper-Scissors game described
in class.

解答:石头剪刀布的双矩阵如下

\begin{tabular}{c|c|c|c}
\hline
&石头&剪刀&布\\
\hline
石头&(0,0)&(1,-1)&(-1,1)\\
\hline
剪刀&(-1,1)&(0,0)&(1,-1)\\
\hline
布&(1,-1)&(-1,1)&(0,0)\\
\hline
\end{tabular}

混合策略唯一纳什均衡解为$(\frac{1}{3},\frac{1}{3},\frac{1}{3})$

当两个人进行石头剪刀布时,如果双方策略均为混合策略,假设纳什均衡解为$(x,y,z)$并有$x+y+z=1$,当一方采取这样
的策略时,另一方为了最大化自己的收益,将以$(y,z,x)$的策略进行游戏,则采用$(x,y,z)$策略的一方期望收益为
\[0*xy+xz-x^2-y^2+0*yz+xy+yz-z^2+0*xz=xy+xz+yz-(x^2+y^2+z^2)\]
由$(x+y+z)^2=x^2+y^2+z^2+2xy+2xz+2yz=1$可知收益为$3*(xy+xz+yz)-1$由基本不等式\[ab\leq(\frac{a+b}{2})^2\]
可知当$x=y=z$时收益有最大值0,此时纳什均衡解为$(\frac{1}{3},\frac{1}{3},\frac{1}{3})$,此时另一方也会调整
策略最大化收益,此时双方策略相同,纳什均衡解唯一

\section*{Lecture 2}

\section*{Problem 1}
\end{document}
